\documentclass{article}
\usepackage{hyperref}
\usepackage{tabularx}
\usepackage{booktabs}
\usepackage[final]{microtype}

\title{COMSM0102 Systems \& Software Security Coursework}
\author{Joseph Hallett and Sana Belguith}
\date\today

\newcommand\releasedate[0]{Friday 8\textsuperscript{th} November}
\newcommand\duedate[0]{Friday 29\textsuperscript{th} November}
\newcommand\cwyear[0]{2024}

\begin{document}
\maketitle
\section*{Part A---Unit Specific Coursework Details}

\begin{description}
\item[Unit Number:] COMSM0102
\item[Unit Name:] Systems \& Software Security
\item[Unit Director:] Sana Belguith
\item[Assignment Name:] Systems \& Software Security Coursework
\item[Assignment Weighting:] 30\%
\end{description}

\subsection*{Assignment Description:}

This coursework is for unit COMSM0102 Systems \& Software Security
Coursework. It will be released on \emph{\releasedate} and must be
submitted by \emph{\duedate} at 1pm at the very latest. The
intention is that you submit by 12pm and keep the last hour as an
emergency reserve for technical problems.

In case of problems with your submission, you must e-mail
\href{mailto://coms-info@bristol.ac.uk}{coms-info@bristol.ac.uk}
before the 1pm final deadline to avoid your work being counted as
late. You must submit the coursework on the Blackboard page for the
assessment unit COMSM0102 (not the teaching unit COMSM0049). On the
assessment unit, go to the menu item \emph{``Assessment, Submission and
  Feedback"} and follow the instructions there.

If you have any questions, please write your question in the Teams
channel and we will answer it as quickly as possible. You are
recommended to do this work in groups of 3 students, but you may work
individually if you prefer. We will consider the group size when we mark
your work. A marking scheme for this coursework is included on the last
page of this document

\subsection*{Introduction}

This project validates the whole unit, there is no other
assessment. It represents a significant investment of time and effort
that should mostly take place during Week 8, 9 and 10. We encourage
you to form groups of 3 students, however you can undertake this
coursework individually if you prefer. You will need as part of the
project to submit: your code, a video demo and a final write up.

We expect every member of a group to participate fully in the project.
You are free to organise as you wish, but your personal contribution
will be evaluated and need to be demonstrated (see below). We are
expecting you to work together and to collaborate effectively. If you
have any concern about your group dynamic or you are struggling to
form a group, do contact us via e-mail and we will be able to help.

If you decide to undertake this coursework individually, we will take
into consideration that you are working alone and we will reflect that
in the marking scheme.

\subsection*{Deliverables}

\begin{description}
\item[Project demonstration (group, graded 30\%)]

  You will demonstrate that your solution works and demonstrate your
  project. Your project should be coming with a \texttt{README}. You will follow
  the \texttt{README} instructions and demonstrate that you obtain the results
  presented in your report and that you can reproduce the evaluation. The
  video should be \emph{no longer than 10 minutes}.

  PS: Please use appropriate compression settings while processing your
  demonstration videos as Blackboard will not allow videos larger than
  100MB to be uploaded.
  README
\item[Final write up (group, graded 70\%)]

  You should submit a report 
  of roughly 5 pages (excluding references and appendice). Your report
  should contain a minimum of six academic citations. We suggest the
  following structure:
  \begin{itemize}
  \item Introduction
  \item Background
  \item Design \& Implementation
  \item Evaluation
  \item Conclusion
  \end{itemize}

  Please use figures to illustrate your points (as appropriate).

  In addition in the appendix you should:
  \begin{enumerate}
  \item Discuss how well you met your project's objective. If
    you did not implement everything you planned this does not mean that you
    will fail (or get a bad grade). You should discuss why it could not be
    done (e.g. technical challenges, change of direction, alternative
    approach taken, sickness of one of the group members etc.).

  \item Your individual contribution to the project as a score (e.g., in
    a group of three if you all worked equally 33\% each) and list your
    individual contributions. You need to all agree on this section.
  \end{enumerate}

  You are all expected to participate in the technical aspects as well as
  the writing. We will take into consideration the complexity of your work
  as well as your individual contributions when deciding your individual
  grades. Our intent is to ensure that no one is penalized if one (or
  several) of the students want to work above and beyond expectations. We
  will only \emph{improve} individual grades, we won't
  award any grades below the report grade.
\end{description}

\subsection*{Project}

In the labs we have been generating exploits using \emph{return oriented programming (ROP)}. For this coursework
takes the lab to the next step by generating such exploits
automatically. In particular:

\begin{enumerate}
\item We assume a stack overflow based vulnerability that overwrites the
  saved return address. You are supposed to automatically find the input
  (string) length that is sufficient to overwrite the saved return address (in the
  lab, you did so by doing manual analysis to find that you needed 44
  bytes of junk data before starting to overwrite the saved return address).

\item In the lab, we were generating a ROP chain thatused to setup:

  \texttt{execve("/tmp//nc","-lnp","5678","-tte","\textbackslash bin//sh",
    NULL)}

  In this courswork, we need to automatically generate the exploit
  which takes arbitrary command line for execve and on a successful
  exploit, you should get that program (argument to \texttt{execve})
  launched. (look at the code of the ROPGadget tool)

\item You need to make sure that the exploit works for any chosen .data
  address (remember, no null bytes!).

\item Rather than forming a ROP of step 2 above (i.e.~arbitrary arguments
  to execve), generate a ROP based exploit for a given arbitrary shellcode
  (see: 
  \href{https://ietresearch.onlinelibrary.wiley.com/doi/epdf/10.1049/iet-ifs.2018.5386}{Transforming Malicious Code to ROP Gadgets for Antivirus
    Evasion})
\end{enumerate}

\subsection*{Support provided to students during coursework period}

Joseph and Sana are both available to answer questions on Teams or by
email, we also set up Q\&A sessions-check your timetables.

\subsection*{Submission Details}

Both final writeup and project
demonstration to be submitted in Blackboard: Systems and Software
Security (with Coursework) \cwyear{} $\rightarrow$
Assessment, submission and feedback Project
$\rightarrow$ Systems and Software Security Coursework

\subsection*{Marking Criteria}

\subsubsection*{Project Demonstration Marking Scheme (30\%)}

\begin{tabularx}{\linewidth}{llX}
  \toprule
  Weight & Category & Comment \\
  \midrule
  10\% & Technical Clarity & You will explain clearly how to run your
                             project. Technical terminology should be used appropriately. You should
                             assume an educated audience of your peers and explain terminologies and
                             concepts specific to your project. It should be clear how that relates
                             to the evaluation section of your report. \\
  20\% & Instructions Clarity & The instructions contained in your \texttt{README}
                                should be simple to follow and lead to the results presented in your
                                report. You need to demonstrate this, by following the instructions step
                                by step in your video. We invite you to start from a "clean"
                                environment. \\
  \bottomrule
\end{tabularx}

\subsubsection*{Report Marking Scheme (70\%)}

\begin{tabularx}{\linewidth}{llX}
  \toprule
  Weight & Category & Comment \\
  \midrule
  10\% & Presentation & You should use the provided latex template
                        properly. Reference should be appropriately formatted. We expect the
                        presentation standard to be on par with the reading material seen during
                        lectures. \\
  20\% & Literature Review & You will identify the relevant academic
                             literature, show understanding of the papers you have selected and cite
                             them appropriately. It should be clear how they relate to your work. You
                             are expected to explore beyond the papers assigned as reading
                             material. \\
  20\% & Design \& Implementation & You should describe your
                                    implementation at an appropriate level of abstraction (refer to the
                                    reading material seen during teaching). You should clearly describe any
                                    technical challenges you faced and articulate the design decisions you
                                    made and why you believe they were appropriate. This should be
                                    understandable by an audience of your peers. \\
  20\% & Evaluation & You should evaluate how well the outcome of your
                      work addresses your objectives. You should use quantitative
                      (e.g.~measuring performance overhead of a security mechanism) or
                      qualitative (e.g. critical discussion of the security guarantees of a
                      mechanism) as appropriate to your project. You are strongly encouraged
                      to draw from evaluations found in the literature to design yours
                      (reference this clearly when this is the case). \\
  \bottomrule
\end{tabularx}

\newpage{}
\section*{Part B---Universal Coursework Details}

\subsection*{Deadline}

The deadline for submission of all optional unit assignments is 13:00 on
\duedate{}. Students should submit all
required materials to the \emph{``Assessment, submission and feedback''}
section of Blackboard---it is essential that this is done on the
Blackboard page related to the \emph{``With Coursework''} variant of the unit.

\subsection*{Time commitment}

You are expected to work on both of your optional unit courseworks in
the 3-week coursework period as if it were a working week in a regular
job---that is 5 days a week for no more than 8 hours a day. The effort
spent on the assignment for each unit should be approximately equal,
being roughly equivalent to 1.5 working weeks each. It is up to you how
you distribute your time and workload between the two units within those
constraints.

You are strongly advised NOT to try and work excessive hours during the
coursework period: this is more likely to make your health worse than to
make your marks better. If you need further pastoral/mental health
support, please talk to your personal tutor, a senior tutor, or the
university wellbeing service.

\subsection*{Academic Offences}

Academic offences (including submission of work that is not your own,
falsification of data/evidence or the use of materials without
appropriate referencing) are all taken very seriously by the University.
Suspected offences will be dealt with in accordance with the
University's policies and procedures. If an academic offence is
suspected in your work, you will be asked to attend an interview with
senior members of the school, where you will be given the opportunity to
defend your work. The plagiarism panel are able to apply a range of
penalties, depending the severity of the offence. These include:
requirement to resubmit work, capping of grades and the award of no mark
for an element of assessment.

\subsection*{Extenuating circumstances}

If the completion of your assignment has been significantly disrupted by
serious health conditions, personal problems, periods of quarantine, or
other similar issues, you may be able to apply for consideration of
extenuating circumstances (in accordance with the normal university
policy and processes). Students should apply for consideration of
extenuating circumstances as soon as possible when the problem occurs,
using the \href{https://www.bristol.ac.uk/request-extenuating-circumstances-form}{extenuating circumstances form}.

You should note however that extensions are not possible for optional
unit assignments. If your application for extenuating circumstances is
successful, it is most likely that you will be required to retake the
assessment of the unit at the next available opportunity.

\end{document}
